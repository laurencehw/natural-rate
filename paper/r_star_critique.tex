\documentclass[12pt,letterpaper]{article}

% Packages
\usepackage[utf8]{inputenc}
\usepackage[margin=1in]{geometry}
\usepackage{amsmath,amssymb,amsthm}
\usepackage{graphicx}
\usepackage{booktabs}
\usepackage{natbib}
\usepackage{hyperref}
\usepackage{setspace}
\usepackage{float}
\usepackage{caption}
\usepackage{subcaption}
\usepackage{threeparttable}
\usepackage{appendix}

% Document settings
\doublespacing
\hypersetup{colorlinks=true, linkcolor=blue, citecolor=blue, urlcolor=blue}

% Title information
\title{Rethinking the Natural Rate of Interest: \\
A Critique of the Laubach-Williams Model\thanks{We thank [acknowledgments]. The views expressed are those of the authors and do not necessarily reflect those of any institution.}}

\author{[Author Name]\thanks{[Affiliation and contact information]}}

\date{\today}

\begin{document}

\maketitle

\begin{abstract}
The Laubach-Williams (LW) and Holston-Laubach-Williams (HLW) models have become the standard approach for estimating the natural rate of interest ($r^*$) for monetary policy purposes. This paper provides a critical assessment of the economic assumptions underlying these models. We identify three key areas of concern: (1) the IS curve specification ignores financial conditions beyond the policy rate; (2) the Phillips curve slope has flattened substantially, calling into question a binding constraint in the model; and (3) the assumption that $r^*$ scales linearly with trend growth ignores the direct role of demographics and other structural factors. We conduct sensitivity analysis showing that $r^*$ estimates are sensitive to constraint choices and sample periods. Alternative specifications incorporating financial conditions and demographics suggest the standard model may understate both the level and uncertainty of $r^*$ estimates. Our findings have important implications for monetary policy frameworks that rely on $r^*$ as a guidepost.

\vspace{0.5cm}
\noindent\textbf{Keywords:} Natural rate of interest, r-star, Laubach-Williams model, monetary policy, secular stagnation, Phillips curve

\noindent\textbf{JEL Codes:} E43, E52, E58, C32
\end{abstract}

\newpage

\section{Introduction}

The natural rate of interest---the real interest rate consistent with output at potential and stable inflation---has become a central concept in modern monetary policy. Since the seminal work of \citet{laubach2003measuring}, policymakers at major central banks have relied on model-based estimates of $r^*$ to calibrate the stance of monetary policy and assess whether current policy rates are stimulative or restrictive.

The Laubach-Williams (LW) model, and its international extension by \citet{holston2017measuring} (HLW), represent the workhorse approach to $r^*$ estimation. The Federal Reserve Bank of New York publishes regular updates to these estimates, which receive significant attention from market participants and policymakers alike. The model's influence extends beyond the United States: similar methodologies have been adopted by the European Central Bank, Bank of Canada, and other institutions.

Despite this widespread adoption, several economic assumptions embedded in the LW/HLW framework deserve scrutiny. This paper provides a systematic critique of these assumptions and examines how alternative specifications affect $r^*$ estimates. We focus on three key areas:

First, the IS curve in the LW model assumes that the real interest rate is the sole financial variable affecting output dynamics. This ignores the substantial literature on the bank lending channel, financial accelerator effects, and the role of financial conditions more broadly in the monetary transmission mechanism \citep{bernanke1995inside, adrian2010financial}.

Second, the model imposes a lower bound constraint on the Phillips curve slope ($b_y > 0.025$). However, there is overwhelming evidence that the Phillips curve has ``flattened'' substantially over recent decades \citep{blanchard2016inflation, hooper2020prospects}. If the true slope is below this constraint, the model may generate biased estimates of both the output gap and $r^*$.

Third, the assumption that $r^*$ scales linearly with trend output growth ($r^* = c \cdot g + z$) ignores the direct effects of demographic changes, inequality, and safe asset shortages that figure prominently in the secular stagnation literature \citep{summers2014us, rachel2017global, carvalho2016demographics}.

Our analysis reveals that $r^*$ estimates are quite sensitive to constraint choices and sample periods. Alternative specifications that incorporate financial conditions or demographics provide equally good or better fit while yielding different implications for the current level of $r^*$. These findings underscore the substantial uncertainty surrounding $r^*$ estimates and caution against over-reliance on any single model.

The remainder of the paper is organized as follows. Section 2 describes the LW/HLW framework and its key identifying assumptions. Section 3 develops our critique of the economic assumptions. Section 4 presents sensitivity analysis and alternative specifications. Section 5 concludes with implications for monetary policy.


\section{The Laubach-Williams Framework}

\subsection{Model Structure}

The LW model estimates $r^*$ through a state-space framework with three core equations: an IS curve relating the output gap to real interest rates, a Phillips curve linking inflation to the output gap, and a law of motion for unobserved states including potential output and the natural rate.

The observation equations are:
\begin{align}
\tilde{y}_t &= a_1 \tilde{y}_{t-1} + a_2 \tilde{y}_{t-2} + \frac{a_r}{2}(r_{t-1} + r_{t-2} - r^*_{t-1} - r^*_{t-2}) + \epsilon^y_t \label{eq:is} \\
\pi_t &= b_1 \pi_{t-1} + b_2 \bar{\pi}_{t-2:t-4} + (1-b_1-b_2)\bar{\pi}_{t-5:t-8} + b_y \tilde{y}_{t-1} + b_o \pi^{oil}_t + b_m \pi^{import}_t + \epsilon^\pi_t \label{eq:phillips}
\end{align}

where $\tilde{y}_t = 100(y_t - y^*_t)$ is the output gap, $r_t$ is the ex ante real interest rate (nominal rate minus expected inflation), $\pi_t$ is inflation, and the $\bar{\pi}$ terms represent moving averages of lagged inflation.

The state equations describe the evolution of potential output and $r^*$:
\begin{align}
y^*_t &= y^*_{t-1} + g_{t-1} + \epsilon^{y^*}_t \label{eq:potential} \\
g_t &= g_{t-1} + \epsilon^g_t \label{eq:trend_growth} \\
z_t &= z_{t-1} + \epsilon^z_t \label{eq:z}
\end{align}

where $g_t$ is trend output growth and $z_t$ captures other factors affecting $r^*$. The natural rate is defined as:
\begin{equation}
r^*_t = c \cdot g_t + z_t \label{eq:rstar}
\end{equation}

\subsection{Estimation Procedure}

The model is estimated via maximum likelihood using the Kalman filter. A key challenge is the ``pile-up problem'': maximum likelihood tends to set the variances of the state innovations ($\sigma_g$, $\sigma_z$) to zero, implying constant $r^*$. Following \citet{stock1998median}, LW use a median-unbiased estimator to recover the signal-to-noise ratios $\lambda_g = \sigma_g/\sigma_{y^*}$ and $\lambda_z$.

The estimation proceeds in three stages:
\begin{enumerate}
\item \textbf{Stage 1}: Estimate potential output ignoring the interest rate channel (setting $a_r = 0$)
\item \textbf{Stage 2}: Add the interest rate to the IS curve; estimate $\lambda_g$ from Stage 1 residuals
\item \textbf{Stage 3}: Full model with $r^*$; estimate $\lambda_z$ from Stage 2 residuals
\end{enumerate}

\subsection{Key Identifying Assumptions}

Several constraints are imposed to ensure economically sensible results:
\begin{itemize}
\item $a_r < -0.0025$: The IS curve slope must be sufficiently negative (interest rates affect output)
\item $b_y > 0.025$: The Phillips curve slope must be sufficiently positive (output gap affects inflation)
\end{itemize}

These constraints play a crucial role in identification. Without them, the model often fails to find statistically significant effects of interest rates on output, making $r^*$ unidentified.


\section{Critique of Economic Assumptions}

\subsection{IS Curve: Beyond the Policy Rate}

\subsubsection{The Omitted Financial Conditions Problem}

The IS curve in equation \eqref{eq:is} assumes that monetary policy affects output solely through the real interest rate. This ``interest rate channel'' view of monetary transmission is increasingly seen as incomplete. A substantial literature documents additional channels:

\textbf{Bank Lending Channel}: Changes in policy rates affect bank reserve positions and willingness to lend, with effects beyond what borrowing costs alone would imply \citep{bernanke1995inside, kashyap2000credit}.

\textbf{Balance Sheet Channel}: Policy rates affect asset prices, collateral values, and balance sheet positions, amplifying the effects on spending \citep{bernanke1999financial}.

\textbf{Risk-Taking Channel}: Low interest rates may encourage financial institutions to ``reach for yield,'' affecting credit supply and risk premia \citep{adrian2010financial, borio2014monetary}.

These channels suggest that financial conditions broadly---not just the policy rate---matter for output dynamics. The Federal Reserve's own National Financial Conditions Index (NFCI) is designed to capture these broader credit and financial market conditions.

\subsubsection{Empirical Evidence}

To assess whether financial conditions add explanatory power beyond the real rate, we augment the IS curve:
\begin{equation}
\tilde{y}_t = a_1 \tilde{y}_{t-1} + a_2 \tilde{y}_{t-2} + a_r \bar{r}_{t-1:t-2} + a_{FCI} \text{NFCI}_{t-1} + \epsilon_t
\end{equation}

Table \ref{tab:fci_results} reports the results. Adding the NFCI significantly improves model fit and, importantly, reduces the estimated coefficient on the real interest rate. This suggests that part of what appears to be an interest rate effect in the baseline model is actually capturing correlated movements in broader financial conditions.

The implications for $r^*$ are significant. If the policy rate is less important for output dynamics than the LW model assumes, the standard estimates may overstate the precision with which we can identify $r^*$.

\subsection{Phillips Curve Flattening}

\subsubsection{The Constraint Problem}

The LW model constrains the Phillips curve slope to be at least 0.025. This constraint is binding in many specifications, particularly in recent data. Figure \ref{fig:phillips_rolling} shows rolling 15-year estimates of the Phillips curve slope. The estimated slope has declined substantially since the 1990s, with recent estimates often below the constraint.

When the constraint binds, the model is forced to attribute inflation dynamics to output gap movements that may not reflect the true relationship. This can distort both the output gap estimate and, through the interconnected system, the $r^*$ estimate.

\subsubsection{Time-Varying Estimates}

We estimate the Phillips curve with a time-varying slope using rolling windows. Key findings:

\begin{itemize}
\item The average slope has declined from approximately 0.15 in the 1970s-80s to near zero in recent decades
\item In more than 40\% of post-2000 rolling windows, the estimated slope is below the 0.025 constraint
\item The decline accelerated following the Great Recession
\end{itemize}

Several explanations have been proposed for Phillips curve flattening: better-anchored inflation expectations \citep{blanchard2016inflation}, globalization and import competition \citep{carney2017phillips}, and changes in labor market structure \citep{hooper2020prospects}.

\subsubsection{Implications for $r^*$}

We re-estimate the LW model with different Phillips curve constraints. Table \ref{tab:by_sensitivity} shows that relaxing the constraint (or eliminating it) leads to somewhat higher recent $r^*$ estimates. The intuition is that with a flatter Phillips curve, less of the decline in inflation is attributed to below-potential output, implying a smaller negative output gap and hence less downward pressure on $r^*$.

\subsection{The $r^* = c \cdot g + z$ Decomposition}

\subsubsection{Theoretical Concerns}

The assumption that $r^*$ scales linearly with trend growth has intuitive appeal: faster productivity growth implies higher investment returns. However, this formulation may be overly restrictive. In standard growth models, the relationship between $r^*$ and growth depends on the source of growth (TFP vs. capital accumulation), preferences, and demographics.

The secular stagnation literature emphasizes factors that affect $r^*$ independently of growth:

\textbf{Demographics}: Aging populations save more for retirement, increasing the supply of savings and pushing down rates \citep{carvalho2016demographics, gagnon2021demographic}.

\textbf{Inequality}: Rising wealth concentration may increase aggregate savings since the wealthy have higher saving rates \citep{mian2021indebted}.

\textbf{Safe Asset Shortage}: Increased demand for safe assets relative to supply depresses safe yields \citep{caballero2017safe}.

\subsubsection{Adding Demographics}

We augment the $r^*$ equation with demographic variables:
\begin{equation}
r^*_t = c \cdot g_t + d_1 \cdot \text{OldAgeDep}_t + d_2 \cdot \text{WAPopGrowth}_t + z_t
\end{equation}

Table \ref{tab:demo_results} presents regression results. Key findings:

\begin{enumerate}
\item The old-age dependency ratio has a significant negative effect on $r^*$
\item Including demographics reduces the coefficient on trend growth and improves model fit
\item Demographics alone explain a substantial portion of $r^*$ variation
\end{enumerate}

These results suggest that the standard LW decomposition conflates demographic effects with growth effects. As populations age, the model may attribute the resulting decline in $r^*$ to slower growth when demographics are the more direct cause.


\section{Sensitivity Analysis and Alternative Specifications}

\subsection{Constraint Sensitivity}

Table \ref{tab:sensitivity} reports $r^*$ estimates under alternative constraint values. Key findings:

\textbf{IS Curve Slope}: Relaxing the constraint (making $a_r$ less negative) tends to raise $r^*$ estimates slightly. When fully unconstrained, the model sometimes fails to find a significant interest rate effect.

\textbf{Phillips Curve Slope}: Relaxing the constraint tends to raise $r^*$ estimates, particularly in recent years. The difference can be economically meaningful (0.3-0.5 percentage points).

\subsection{Sample Period Sensitivity}

We estimate the model over different sample periods:

\begin{itemize}
\item \textbf{Full sample (1961-present)}: Baseline specification
\item \textbf{Post-Volcker (1985-present)}: Excludes the Great Inflation
\item \textbf{Post-GFC (2010-present)}: Post-crisis period only
\end{itemize}

Results vary substantially across samples. The full sample estimates are strongly influenced by the 1960s-70s period when the Phillips curve was steeper and interest rate effects may have differed. Post-1985 estimates tend to show a flatter Phillips curve and less certain identification of $r^*$.

\subsection{COVID Period Treatment}

The COVID-19 pandemic poses challenges for estimation. The official LW/HLW code addresses this through:
\begin{itemize}
\item Variance scaling factors ($\kappa$) to allow for larger shocks during 2020-2022
\item A COVID indicator variable in the output gap equation
\end{itemize}

We examine sensitivity to COVID treatment. Ending the sample in 2019Q4 (pre-COVID) versus including COVID data with the adjustments affects both the level and uncertainty of recent $r^*$ estimates.

\subsection{US-Euro Area Comparison}

Figure \ref{fig:us_ea} compares $r^*$ estimates for the United States and Euro Area using the HLW model. Both series show similar trends: high values in the 1960s-70s, decline through the 1980s-90s, and low (sometimes negative) values since the GFC.

However, the Euro Area estimates show greater volatility, reflecting both shorter sample history and different Phillips curve dynamics. The correlation between US and EA $r^*$ is high, suggesting common global factors at work.


\section{Conclusion}

The Laubach-Williams model has become the standard approach for estimating the natural rate of interest. However, several economic assumptions embedded in the model deserve scrutiny.

Our analysis reveals three key concerns:

\begin{enumerate}
\item The IS curve ignores financial conditions beyond the policy rate. Adding financial conditions improves model fit and reduces the estimated policy rate effect, suggesting the standard model may overstate how precisely we can identify $r^*$.

\item The Phillips curve slope constraint may be binding inappropriately in recent data. Rolling estimates suggest the true slope is often below the constraint, potentially distorting both output gap and $r^*$ estimates.

\item The $r^*$ decomposition ignores demographics and other structural factors emphasized in the secular stagnation literature. Including demographics improves fit and changes the interpretation of $r^*$ movements.
\end{enumerate}

These findings have important implications for monetary policy. If $r^*$ estimates are as uncertain as our analysis suggests, policymakers should be cautious about fine-tuning policy based on these estimates. The confidence intervals published with the official estimates may understate true uncertainty.

More broadly, our results suggest that the decline in $r^*$ over recent decades may reflect structural factors---particularly demographics---that are unlikely to reverse quickly. This has implications for the conduct of monetary policy in a persistently low-rate environment.

Future research could extend this analysis in several directions: formal Bayesian estimation with alternative priors, explicit modeling of financial conditions in the state-space framework, and international comparisons that exploit cross-country variation in demographics and financial development.


\newpage
\bibliographystyle{apalike}
\bibliography{references}

\newpage
\appendix

\section{Data Sources and Definitions}

\begin{itemize}
\item \textbf{GDP}: Real gross domestic product, billions of chained 2017 dollars, seasonally adjusted annual rate (FRED: GDPC1)
\item \textbf{Inflation}: Personal consumption expenditures price index, annualized quarterly change (FRED: PCEPI)
\item \textbf{Federal Funds Rate}: Effective federal funds rate (FRED: FEDFUNDS)
\item \textbf{Inflation Expectations}: University of Michigan inflation expectations (FRED: MICH)
\item \textbf{NFCI}: Chicago Fed National Financial Conditions Index (FRED: NFCI)
\item \textbf{Old-Age Dependency Ratio}: Population 65+ / Population 15-64 (OECD)
\end{itemize}


\section{Replication Code}

All code for this paper is available at [repository URL]. The analysis uses the official LW/HLW replication code from the Federal Reserve Bank of New York, modified to accommodate our alternative specifications.


\end{document}
